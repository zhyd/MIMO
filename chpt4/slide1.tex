\documentclass[11pt]{beamer}
\usetheme{Pittsburgh}
\usepackage[utf8]{inputenc}
\usepackage{amsmath}
\usepackage{amsfonts}
\usepackage{amssymb}
%\author{Author}
\title{RF Propogation}
%\setbeamercovered{transparent} 
%\setbeamertemplate{navigation symbols}{} 
%\logo{} 
\institute{BITS Pilani} 
%\date{} 
\subject{RF Propogation} 
\begin{document}

\begin{frame}
\maketitle
\end{frame}
\begin{frame}{Outline}
  \tableofcontents
\end{frame}

% Section and subsections will appear in the presentation overview
% and table of contents.
\section{First Main Section}

\subsection{First Subsection}

\begin{frame}{Delay Spread}
\begin{itemize}
\item Range of propagation delays
\item Autocorrelation of the channel impulse response $c(t,\tau)$
\item Can assume channel is Wide Sense Stationary - $c(t,\tau)$ depends on $\Delta t$ [p95]
\item Attenuation and phase shift with $\tau 1$ uncorrelated with $\tau 2$
\item $E_r$ multipath intensity profile - transmit impulse - measure received power as function of time [f4.9/4.10]
\item Inference 
\begin{enumerate}
\item weaker power as delay increases  
\item bumps in delay profile - scatterers not uniform in space 
\end{enumerate}
 \item Delay spread can be calculated without knowing the absolute received power levels
\item Commercial wireless devices scattering insensitive to an extent to RF frequencies
\end{itemize}
\end{frame}
\begin{frame}{Coherence Bandwidth}
\begin{itemize}
\item Correlation between channel outputs given the input sinusoids are separated in frequency by $\Delta f$
\end{itemize}
\end{frame}
\begin{frame}{Doppler Model}
\begin{itemize}
\item Clark doppler model[p103] used in cellular standards like LTE, LTE-A, WiMAX
\item WLAN channel dynamics, transmitter and receiver fixed - Clark not used [p104]
\item Coherence time - max time between two signals - max correlation when received(channel response is same)
\end{itemize}
\end{frame}

\begin{frame}
\begin{itemize}
\item flat fading [p106] - simple multiplication of the transmit signal by a complex gain
\item Rician fading - recevied signal - fluctuating(scattering) part and non-fluctuating(direct) 
\end{itemize}

\end{frame}

\begin{frame}
\textbf{Parameters related to tau}
delay spread and coherence bandwidth . \\
time dispersion refers to spreading a signal out in the $\tau$ time dimension due to multipath. these two parameters do not provide information about the time-varying nature of the channel. 
\end{frame}

\end{document}